\chapter{Protocol Specifications}
	
	The printer provides two network interfaces for which the protocols are
	described in the following sections.
	
	The port numbers used by these services are shown in table \ref{tab:ports}.
	
	\begin{table}
		\centering
		\begin{tabular}{l l}
			\toprule
			Port & Service \\
			\midrule
			1818 & G-code interface (UDP \& TCP) \\
			2777 & Status interface (TCP) \\
			\bottomrule
		\end{tabular}
		
		\caption{Network interface ports}
		\label{tab:ports}
	\end{table}
	
	\section{UDP G-code Transmission Protocol}
		
		\label{sec:udpSpec}
		
		This interface is required to reliably send data to the printer and provide
		flow control such that the printer doesn't receive data it can't handle. A
		discussion of the protocol's general design and requirements can be found in
		\S\ref{sec:udpimpl}
		
		\subsection{Sender Packets}
			
			Packets sent to the printer consist of an unsigned 32-bit integer sequence
			number followed by bytes of payload data up to the window size advertised
			by the receiver.
			
			Initially the sequence number is 0 and the window size is assumed to be
			zero. This means that the first datagram consists of the sequence number
			alone.
			
		\subsection{Acknowledges}
			
			The receiver is required to acknowledge each datagram received and
			accepted and advertise its current window size. Packets consist of a copy
			of the 32-bit sequence number followed by an unsigned 32-bit integer
			window size.
			
			Datagrams are only accepted when
			\begin{itemize}
				
				\item Their sequence numbers are sequential
				
				\item An empty first packet with sequence number zero has been received
				
				\item The data contained fits into the system buffer (i.e. obeys the
				window size)
				
			\end{itemize}
			
		\subsection{Retransmission}
			
			If the sender does not receive an acknowledgement with a matching sequence
			number within a certain period it will repeatedly resend the request.
			
			If a zero-window size is reported, the sender must poll the printer until
			the window re-opens by issuing new empty packets with increasing sequence
			numbers.
			
			Exponential back-off is not used in an effort to reduce the latency
			between the window opening and sending the next piece of data. Shorter
			timeouts result in more traffic and less delay in restarting transmission
			after a zero window. Longer timeouts result in less wasted traffic but, if
			too long, can result in buffer underruns before more data is sent after a
			zero window.
			
		\subsection{Errors}
			
			If the printer receives some unexpected data (for example if the printer
			is reset and receives a datagram from the middle of a transmission before
			it was reset), an acknowledgement with sequence number 0 and a zero-window
			is returned indicating a terminal error.
			
	\section{Status Monitoring Protocol}
		
		\label{sec:statusSpec}
		
		To monitor the printer's status, newline-terminated commands are sent to the
		printer over a TCP connection. Responses consist of a series of integers
		separated by spaces and terminated with a newline. Table
		\ref{tab:statuscmds} lists the commands and their responses.
		
		The `Integer $\times100$' format is used for decimal values as printing
		floating point values are not supported by the standard libraries used. The
		floating point value is multiplied by 100 and truncated to an integer.
		
		\begin{table}
			\centering
			\begin{tabular}{l l l l}
				\toprule
				Command & Response & Unit & Format \\
				\midrule
				\verb|tmp| & Extruder temperature & \dC & Integer $\times100$ \\
				           & Extruder set point   & \dC & Integer $\times100$ \\
				           & Extruder heater on   &     & Boolean \\
				           & Platform temperature & \dC & Integer $\times100$ \\
				           & Platform set point   & \dC & Integer $\times100$ \\
				           & Platform heater on   &     & Boolean \\
				\addlinespace
				\verb|gcd| & G-code instructions interpreted & & Integer \\
				           & Last interpreter error number &  & Integer \\
				\addlinespace
				\verb|buf| & G-code buffer utilisation & Bytes & Integer \\
				           & G-code buffer size & Bytes & Integer \\
				           & Command buffer utilisation & Commands & Integer \\
				           & Command buffer size & Commands & Integer \\
				           & Command buffer overrun count this print & & Integer \\
				\addlinespace
				\verb|pow| & PSU Power On & & Boolean \\
				\addlinespace
				\verb|pos| & X-axis Position & Millimetres & Integer $\times100$ \\
				           & Y-axis Position & Millimetres & Integer $\times100$ \\
				           & Z-axis Position & Millimetres & Integer $\times100$ \\
				\addlinespace
				\verb|dbg| & Debug Register & Undefined & Undefined \\
				\bottomrule
			\end{tabular}
			
			\caption{Status interface commands and responses}
			\label{tab:statuscmds}
		\end{table}
