Personal 3D printers allow anyone to manufacture complex physical objects on
demand with minimal skill and effort. The Makerbot is a low cost, easy to build
3D printer with a wide community of users. Though relatively capable, the
machine is held back by its primitive control electronics and microcontroller.

New electronics based around a powerful ARM microcontroller were produced
along with a complete 3D printer control system. The system was built on
the widely used FreeRTOS and uses the \uIP{} TCP/IP stack allowing a clean,
easily extended design to be implemented.

The new system improved the print quality achievable by the printer thanks to
improved timing accuracy. Additional sensors also allowed the printer to act
with increased independence.
