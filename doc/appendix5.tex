\chapter{Code Documentation}
	
	This appendix contains the documentation required to configure and build the
	microcontroller firmware. It also contains the required documentation for
	using the communication utilities.
	
	\section{File Listing}
		
		Table \ref{tab:files} describes the purpose of each of the key project
		files.
		
		\begin{table}
			\begin{tabular}{p{0.45\textwidth} p{0.45\textwidth}}
				\toprule
				Filename & Description \\
				\midrule
				\verb|core_cm3.h| & ARM CMSIS header for Cortex-M3 \\
				\verb|cr_startup_lpc17.c| & Defines interrupt-handler functions \\
				\verb|FreeRTOSConfig.h| & System \& networking parameters \\
				\verb|freertos_hooks.{c,h}| & FreeRTOS callback handlers \\
				\verb|LPC17xx.h| & NXP CMSIS header for LPC17xx series \\
				\verb|main.c| & Initialises system starts all tasks \\
				\verb|MakebedConfig.h| & Configuration for 3D printing firmware \\
				\verb|makedefs| & Definitions used by the Makefile \\
				\verb|Makefile| & Makefile which builds the project \\
				\verb|mbed_boot.{c,h}| & Functions to boot up chip peripherals \\
				\verb|rtosdemo_rdb1768_Debug.ld| & Linker script \\
				\verb|system_LPC17xx.h| & ARM CMSIS header for Cortex-M3 \\
				\addlinespace
				\verb|analog/analog_in.{c,h}| & Analog input driver \\
				\addlinespace
				\verb|float_parsing/strtod.{c,h}| & Implementation of strtod \\
				\addlinespace
				\verb|GPIO/gpio.{c,h}| & General purpose input/output driver \\
				\addlinespace
				\verb|makerbot/makerbot.{c,h}| & Print manager/scheduler \\
				\addlinespace
				\verb|network/emac.h| & FreeRTOS EMAC driver \\
				\verb|network/EthDev*.h| & Ethernet driver \\
				\verb|network/network.{c,h}| & Network interface \uIP{} `application' \\
				\verb|network/network_uip_state.h| & State struct for \uIP{} `application' \\
				\verb|network/network_debug.{c,h}| & Debugging/Status interface \\
				\verb|network/network_gcode.{c,h}| & G-code transmission interface \\
				\addlinespace
				\verb|pid/pid.{c,h}| & PID controller \\
				\addlinespace
				\verb|stepper/stepper.{c,h}| & Stepper motor driver \\
				\addlinespace
				\verb|thermistor/thermistor.{c,h}| & Thermistor library \\
				\addlinespace
				\verb|util/makebed.py| & Communication utility \\
				\verb|util/makebed_live.sh| & Live printer status monitor\\
				\verb|util/gsend.py| & G-code UDP library \\
				\verb|util/debug.py| & Status interface library \\
				\verb|util/gcode/*.gcode| & Example G-code files \\
				\addlinespace
				\verb|watchdog/watchdog.{c,h}| & Watchdog timer driver \\
				\bottomrule
			\end{tabular}
			
			\caption{Key source file listing}
			\label{tab:files}
		\end{table}
	
	\section{Firmware}
		
		This section describes the compilation and configuration process for the
		Mbed firmware.
		
		\subsection{Dependencies}
			
			The tools/libraries required to build the firmware are listed in table
			\ref{tab:dependencies}.
			
			\begin{table}
				\centering
				\begin{tabular}{l l}
					\toprule
					Package & Version \\
					\midrule
					CodeSourcery G++ Lite & 2011.03-42 \\
					FreeRTOS & 7.0.2 \\
					GNU Make & 3.81 \\
					\bottomrule
				\end{tabular}
				
				\caption{Firmware software dependencies}
				\label{tab:dependencies}
			\end{table}
			
		\subsection{Compilation}
			
			\label{sec:compilation}
			
			The following steps can be used to build the Mbed firmware:
			
			\begin{enumerate}
				
				\item Make sure the dependencies mentioned above are installed
				
				\item Ensure that \verb|arm-none-eabi-gcc| is in your path
				
				\item In the Makefile, set \verb|RTOS_SOURCE_DIR| to the \verb|Source|
				directory of your FreeRTOS distribution.
				
				\item In the Makefile, set the path of the Mbed mount point on your
				computer in the \verb|install: all| make rule.
				
				\item Run \verb|make| to build the firmware
				
				\item \verb|make install| will build as above and also copy the firmware
				to the Mbed. Reset the Mbed to start the newly installed firmware.
				
			\end{enumerate}
		
		\subsection{Configuration}
			
			All printer-related parameters can be found fully documented in
			\verb|MakebedConfig.h|.
			
			To set the system's IP, net-mask and MAC addresses, the definitions
			\verb|configIP_ADDR0-3|, \verb|configIP_ADDR0-3| and
			\verb|configMAC_ADDR0-5| can be set as the contents of the dot-file
			\verb|FreeRTOSConfig.h|. DHCP is not supported.
			
	
	\section{Utilities}
		
		\label{sec:utilDoc}
		
		Utilities are provided for interacting with the printer over the network.
		Their dependencies and use are described below. They should be executed from
		the \verb|util| directory.
		
		The utilities require that the printer's IP address is specified in
		\verb|~/.makebedrc|.
		
		\subsection{Dependencies}
			
			The utilities depend on the programs mentioned in table
			\ref{tab:utildependencies}.
			
			\begin{table}
				\centering
				\begin{tabular}{l l}
					\toprule
					Package & Version \\
					\midrule
					Python & 2.6 \\
					Bash & 4.1 \\
					GNUPlot & 4.4 patchlevel 0 \\
					\bottomrule
				\end{tabular}
				
				\caption{Utility dependencies}
				\label{tab:utildependencies}
			\end{table}
		
		\subsection{Send}
		
			\label{sec:makebedDoc}
			
			Files can be sent to the printer using the command
			\begin{verbatim}
				makebed.py send
			\end{verbatim}
			which accepts G-code on its standard input or from a file specified as an
			argument. The program will block until the entire file has been received
			by the printer.
			
			If the printer is suffering buffer underruns, the rate at which the
			utility polls the printer can be increased with the \verb|-p| option which
			takes a number of seconds (defaulting to 0.1) to wait between poll
			requests. This should not need changing for most prints and can result in
			a large amount of network traffic being generated if set very low.
			
			\subsubsection{Safety}
				
				If the program is terminated while sending a G-code file, the printer
				will keep running after executing the last intact G-code instruction
				received. During a typical print this could leave the printer stationary
				but extruding plastic indefinitely potentially causing damage or unsafe
				behaviour. As a precaution, it is recommended that the printer is reset
				using the reset button if the utility is stopped prematurely.
		
		\subsection{Get}
			
			The command
			\begin{verbatim}
				makebed.py get
			\end{verbatim}
			allows printer status information to be fetched. The command expects a
			list of categories for which status should be fetched and returns a
			tab-delimited data file containing those values and exits.
			
			The \verb|-p| option causes the printer to be repeatedly polled for status
			information until the utility is terminated.
			
			The output of this utility is compatible with GNU Plot.
		
		\subsection{makebed\_live.sh}
			
			\label{sec:makebedliveDoc}
			
			To provide a live view of the printer's status during a print, a shell
			script has been provided which plots live data from the printer's status
			interface. It is a simple wrapper for the \verb|makebed.sh get| command. A
			screen shot is shown in figure \ref{fig:makebedlive} on page
			\pageref{fig:makebedlive}.
