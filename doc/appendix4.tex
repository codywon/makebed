\chapter{G-Code Reference}
	
	The subset of G-code interpreted by the system is described in the following
	sections.
	
	\section{Language}
		
		The G-code machine is described in \S\ref{sec:gcodemachine}. The following
		subsections specify the syntax and registers of the language.
		
		\subsection{BNF}
			
			\label{sec:gcodebnf}
			
			\begin{verbatim}
				<instruction> ::= <reg-write> <new-line>
				
				<reg-write> ::= <reg-name> <number> <white-space>* <reg-write> | <comment> | ""
				<reg-name>  ::= [A-Z]
				
				<comment>            ::= <line-comment> | <block-comment>
				<line-comment>       ::= <line-comment-start> <non-newline>* <new-line> 
				<line-comment-start> ::= ";" | "/"
				<block-comment>      ::= "(" <non-close-bracket>* ")"
			\end{verbatim}
		
		\subsection{Register Types \& Behaviour}
			
			\begin{table}[H]
				\centering
				\begin{tabular}{c l l}
					\toprule
					Register & Type & Reset at instruction start \\
					\midrule
						A & Integer & No  \\
						B & Float   & No  \\
						C & Float   & No  \\
						D & Float   & No  \\
						E & Float   & No  \\
						F & Float   & No  \\
						G & Integer & Yes \\
						H & Float   & No  \\
						I & Float   & No  \\
						J & Float   & No  \\
						K & Float   & No  \\
						L & Float   & No  \\
						M & Integer & Yes \\
						N & Float   & No  \\
						O & Float   & No  \\
						P & Integer & No  \\
						Q & Float   & No  \\
						R & Float   & No  \\
						S & Float   & No  \\
						T & Integer & No  \\
						U & Float   & No  \\
						V & Float   & No  \\
						W & Float   & No  \\
						X & Float   & No  \\
						Y & Float   & No  \\
						Z & Float   & No  \\
					\bottomrule
				\end{tabular}
				
				\caption{G-code register types and behaviours}
				\label{tab:gcoderegisters}
			\end{table}
	
	\section{Actions}
		
		\label{sec:gcodeactions}
		
		Each instruction should write a value to the `G' or `M' register. Depending
		on the value written, the printer will carry out a diferrent action. These
		actions are specified below.
		
		\newcommand{\gcodeaction}[4]{
			\subsubsection{#1 : #2}
				#3
				
				\begin{table}[H]
					\begin{tabular}{c p{0.41\textwidth} p{0.41\textwidth}}
						\toprule
						Argument & Description & Unit \\
						\midrule
						#4
						\bottomrule
					\end{tabular}
				\end{table}
		}
		\newcommand{\gcodearg}[3]{#1 & #2 & #3 \\}
		\newcommand{\gcodenoargs}{\multicolumn{3}{c}{\emph{No arguments}}\\}
		
		\subsection{`G' Actions}
			
			\gcodeaction{G0 \& G1}{Move extruder to coordinate}{
				G1 moves the extruder through a straight line from the current position
				to the position given in the arguments.
				
				G0 is included for compatibility reasons but does the same thing as G1.
			}{
				\gcodearg{X}{X-coordinate}{Current unit}
				\gcodearg{Y}{Y-coordinate}{Current unit}
				\gcodearg{Z}{Z-coordinate}{Current unit}
				\gcodearg{F}{Feed rate (speed) of movement}{Current units per minute}
			}
			
			\gcodeaction{G4}{Sleep}{
				Pause the printer for a specified period.
			}{
				\gcodearg{P}{Period}{Miliseconds}
			}
			
			\gcodeaction{G20 \& G21}{Set unit}{
				Set the units used to specify movements to inches or milimeters
				respectively. The default is inches for historical reasons.
			}{
				\gcodenoargs
			}
			
			\gcodeaction{G90 \& G91}{Set absolute/relative positioning}{
				Sets whether positions are specified absolutely or relative to the
				current position. Relative positioning is not supported by the system.
			}{
				\gcodenoargs
			}
			
			\gcodeaction{G92}{Set origin}{
				Set the current position without moving the extruder. Used to set the
				initial position of the extruder to the origin or some known homing
				location. If this instruction is not used, the current position is
				undefined and moving the extruder may have unexpected effects.
			}{
				\gcodearg{X}{X-coordinate}{Current unit}
				\gcodearg{Y}{Y-coordinate}{Current unit}
				\gcodearg{Z}{Z-coordinate}{Current unit}
			}
		
		\subsection{`M' Actions}
			
			\gcodeaction{M-2}{Home axes}{
				Custom extension to the standard G-code actions.
				
				Slowly move the specified axes until hitting the endstop. The current
				position in the X, Y and Z registers is then set to the locations of the
				end-stops (essentially calibrating the printer's position).
			}{
				\gcodearg{A}{Axis Selection}{Bit mask. Bit 1: X, Bit 2: Y, Bit 3: Z.}
			}
			
			\gcodeaction{M-1 \& M0}{Turn the PSU on/off}{
				M-1 is a custom extension to the standard G-code actions.
				
				Turns the PSU on and off respectively. Note that if the PSU is not
				powered on, many actions may block indefinately.
				
				This action blocks until mains power is available if the
				microcontroller is being powered via USB.
			}{
				\gcodenoargs
			}
			
			\gcodeaction{M6}{Wait for heaters}{
				Block until both heaters have reached their target temperature. Contrary
				to the standard, this command will not block waiting for the heaters to
				cool down to a new target temperature.
			}{
				\gcodenoargs
			}
			
			\gcodeaction{M17 \& M18}{Power stepper motors on/off}{
				Enable or disable power to all stepper motor drivers. If the stepper
				motors are powered they cannot be moved manually. The steppers are
				automatically powered on when moved.
			}{
				\gcodenoargs
			}
			
			\gcodeaction{M101, M102 \& M103}{Extruder motor forward/backward/off}{
				Set the extruder motor moving forward, backward or not at all
				respectively. M102 is not supported by the system and will raise an
				error and stop the motor motor.
				
				The extruder should not be turned on unless it has heated up enough to
				melt the incoming filament.
			}{
				\gcodenoargs
			}
			
			\gcodeaction{M104}{Set extruder temperature}{
				Set the target temperature of the extruder. This action does not block,
				to wait for heating to complete use M6.
			}{
				\gcodearg{S}{Target temperature}{\dC}
			}
			
			\gcodeaction{M106 \& M107}{Platform conveyor on/off}{
				Turn the platform conveyor belt on or off respectively.
			}{
				\gcodenoargs
			}
			
			\gcodeaction{M108}{Set extruder speed}{
				Set the speed at which the extruder motor turns. Not supported by the
				system.
			}{
				\gcodenoargs
			}
			
			\gcodeaction{M109}{Set platform temperature}{
				Set the target temperature of the platform. This action does not block,
				to wait for heating to complete use M6.
			}{
				\gcodearg{S}{Target temperature}{\dC}
			}
	
	\section{Examples}
		
		The following example G-code files show how G-code can be used for various
		useful tasks.
		
		\subsection{Power On, Heat Up}
			
			Heats the system up to a temperature suitable for printing. Can be used to
			prepare the printer before starting a print.
			
			\begin{verbatim}
				(Turn on the PSU)
				M-1
				
				(Set the target temperature for the extruder to 225*c)
				M104 S225
				
				(Set the target temperature for the platform to 120*c)
				M109 S120
			\end{verbatim}
			
		\subsection{Power-down}
			
			Powers down all components and then the PSU. When the PSU is turned back
			on, the heaters and motors will still remain off.
			
			\begin{verbatim}
				(Turn off extruder)
				M104 S0 (Heater: set target to 0*c)
				M103    (Motor)
				
				(Turn off platform)
				M109 S0 (Heater: set target to 0*c)
				M107    (Conveyor)
				
				(Turn off stepper motors)
				M18
				
				(Power off PSU)
				M0
			\end{verbatim}
		
		\subsection{Skeinforge Print Prefix}
			
			Prifix added to all print jobs to heat up and prepare the printer before a
			print job.
			
			\begin{verbatim}
				(**** begin initilization commands ****)
				
				(power on)
				M-1
				
				G21 (set units to mm)
				G90 (set positioning to absolute)
				
				(Start in parking position)
				G92 X-60 Y-45 Z10
				
				(Raise up to avoid the loop)
				G1 Z12 F100
				
				(Move to squirt position)
				G1 X-55 Y-10 F1000
				G1 Z7 F100
				
				(Heat up)
				M104 S225
				M109 S120
				M6
				
				(Extrude a bit and stop)
				M101
				G4 P5000
				M103
				G4 P6000
				
				(Wipe)
				G1 Y10 F2000
				
				(Go to origin)
				(M101)
				(G1 X0 Y0 Z0 F2400.0)
				
				(**** end initilization commands ****)
			\end{verbatim}
		
		\subsection{Skeinforge Print Postfix}
			
			Postfix added to all print jobs to cool down and eject the object after
			printing.
			
			\begin{verbatim}
				(**** begin ending commands ****)
				
				G1 X0 Y40 F3300.0 (move platform to ejection position)
				(cool down platform)
				M104 S225
				M109 S80
				M103 (Extruder off)
				G04 P100000 (wait t/1000 seconds)
				M106 (conveyor on)
				G04 P10000 (wait t/1000 seconds)
				M107 (conveyor off)
				
				(start wipe)
				(Move to squirt position)
				G1 X-55 Y-10 F1000
				G1 Z7 F100
				
				(Heat up extruder)
				M104 S225
				M6
				
				(Extrude a bit and stop)
				M101
				G4 P5000
				M103
				G4 P6000
				
				(Wipe)
				G1 Y10 F2000
				
				(Go to starting position)
				G1 Z12 F100
				G1 X-60 Y-45 F3300
				G1 Z10 F100
				
				
				(Turn off heaters)
				M104 S0 (set extruder temperature)
				M109 S0 (set heated-build-platform temperature)
				
				(power off)
				M0
				
				(**** end ending commands ****)
			\end{verbatim}
		
		\subsection{Home X \& Y Axes}
			
			Home the X \& Y axes using the end stops. Assumes that the Z-axis starts at
			the correct height to fit in the homing bracket. Raise the nozzle out
			of the homing bracket, home to the endstops and then place the nozzle in
			the calibration bracket.
			
			\begin{verbatim}
				(Power on)
				M-1
				
				(Use mm)
				G21
				
				(Set the Z axis as we're not homing that)
				G92 X0 Y0 Z10
				
				(Lift the head out of its hole)
				G1 Z15 F100
				
				(Home x[1] and y[2] at the same time[1+2 = 3])
				M-2 A3
				
				(Move to the calibration ring)
				G1 X-56 Y-44 Z15 F3300
				G1 Z10 F100
				
				(Power off)
				M0
			\end{verbatim}
			
			\label{sec:gcode_home_xy}
		
		\subsection{Circle}
			
			Plots a circle segmented into lines using the X and Y axes. Assumes the
			extruder is hovering safely above the center of the platform before
			moving.
			
			\begin{verbatim}
				(Power on)
				M-1
				
				(Use mm)
				G21
				
				(Assume we're starting in the middle)
				G92 X0 Y0 Z0
				
				(Move to the edge of the circle)
				G1 X40.000000 Y0.000000 F330.000000
				
				(Plot the circle)
				G1 X32.360680 Y23.511410 F330.000000
				G1 X12.360680 Y38.042261 F330.000000
				G1 X-12.360680 Y38.042261 F330.000000
				G1 X-32.360680 Y23.511410 F330.000000
				G1 X-40.000000 Y0.000000 F330.000000
				G1 X-32.360680 Y-23.511410 F330.000000
				G1 X-12.360680 Y-38.042261 F330.000000
				G1 X12.360680 Y-38.042261 F330.000000
				G1 X32.360680 Y-23.511410 F330.000000
				
				(Move to the center of the circle
				G1 X0 Y0 F330.000000
				
				(Power off)
				M0
			\end{verbatim}

