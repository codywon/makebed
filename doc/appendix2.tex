\chapter{Example Prints}
	
	\label{sec:examplePrints}
	
	\newcommand{\examplePrint}[2]{
		\begin{figure}[H]
			\includegraphics[width=1\textwidth]{diagrams/#1.pdf}
			\caption{#2}
			\label{fig:#1}
		\end{figure}
	}
	
	This appendix shows a representative selection of the objects that were
	printed as test cases during system testing. All designs printed were
	downloaded from Thingiverse or included as part of ReplicatorG.
	
	\section{Intricate Prints}
		
		\examplePrint{snake}{Flexible snake to test printing many thin fins}
		
		\examplePrint{starfish}{Starfish to test layerering appearence}
		
		\examplePrint{doorstop}{Doorstop to test large, smooth gradients}
		
		\examplePrint{butterfly}{Butterfly to test intricate islands of print}
		
		\examplePrint{rabbit}{Rabbit outline to test very thin structures}
		
	\section{Large Prints}
		
		\examplePrint{letter}{Letter `A' to test simple large shapes}
		
		\examplePrint{phoneDock}{Phone stand to test steep gradients}
		
		\examplePrint{tooth}{Tooth to test large models with large overhanging areas}
		
		\examplePrint{joblot}{Multiple `metabricks' to test building batches of objects}
		
	\section{Functional Prints}
		
		\examplePrint{heart}{Twistable heart to test simple mechanisms and
		                     multi-part objects}
		
		\examplePrint{whistle}{Whistle (with pea printed inside) to test precise,
		                       air-tight objects with simultaneously printed
		                       sub-components}
		
		\examplePrint{tweezers}{Tweezers to test flexible designs (frequently used
		                        to remove excess extrusion produced during
		                        self-cleaning)}
