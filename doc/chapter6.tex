\chapter{Conclusions \& Future Work}
	
	\label{sec:conclusions}
	
	In this chapter, the outcomes of the project are compared against the initial
	goals and future work is suggested based on lessons learnt and the limitations
	the project.
	
	\section{Project Goals}
		
		The three project goals have each been addressed within the project and in
		this section the degree to which they have been met is discussed.
		
		\subsection{Electronics Replacement}
			
			The new electronics functionally replace all the features of the old
			electronics and have proved reliable in testing.  With a single board the
			system is also a lot simpler than the previous three boards (removing the
			need for a custom communications protocol and extra microcontroller).
			Though not quite as polished as a printed circuit board (PCB), it offers
			the possibility future expansion.
			
			The solid state MOSFETs, used for heater control, have performed well with
			the silent operation notable over the previous system. They also allow the
			possibility of variable power controls for the motors and heaters with
			only software changes.
			
			The only notable issue remaining is its compatibility with certain ATX
			PSUs. Despite the simplicity of the fix, time was not available to
			implement and test the change. Future work should aim to address this
			issue.
		
		\subsection{Microcontroller Improvements}
			
			The choice of using the Mbed microcontroller proved overall to be a good
			decision. It provided all the hardware required in a small, easily
			integrated package. The device was also fast enough to host the printer
			software providing a major improvement over the old system. Unfortunately,
			the lack of exposed debugging facilities significantly hindered
			development in some areas. For example, with JTAG debugging facilities
			available, adding a new networking stack may have been possible within the
			project time scale.
			
			The firmware written for the Mbed proved to be an improvement over the
			previous system making more detailed prints possible due to improved
			system performance and communications. It is also far more modular in
			design making expansion and experimentation feasible. The print quality
			achieved by the printer is now largely consistent with the constraints of
			the hardware itself rather than inadequacies of the firmware or
			microcontroller.
			
			FreeRTOS proved a good choice of operating system as it provided a
			reliable multi-process environment. It also didn't place any restrictions
			on development allowing straightforward code for interacting with the
			low-level features of the Mbed.
			
			\uIP{}, though the best option within the constraints of the project due
			to its simplicity, wasn't especially well suited to the task. The flow
			control problem greatly diminished the utility of the stack by forcing the
			development of a custom protocol. As well as this, the API it enforces
			(requiring protothreads and protosockets) is likely to make future
			expansion of the network interface difficult due to it's heavy
			restrictions in the name of unneeded performance savings. Future changes
			to overcome these problems are discussed in \S\ref{sec:future_network}.
			
			The stepper and heater control systems developed, though relatively
			na\"ive, proved performant and improved on the old system. Possible future
			improvements are described in \S\ref{sec:future_firmware}.
			
			Overall, the improvements made to the microcontroller and firmware were
			successful and have had a positive effect on the performance and future
			expandability of the printer.
		
		\subsection{End-stops}
			
			Though susceptible to glitches under direct lighting, the end-stops developed
			functioned correctly and allowed automatic calibration of the axis
			positions as required. They also obey the standard interface used by other
			Makerbot end-stops meaning future modifications may make use of standard
			designs.
			
			The need to build a control circuit rather than using pre-made PCBs
			resulted in another circuit board on the printer reducing some of the
			simplicity gained from simplifying the main electronics. The CAT-5 cables
			used to connect the board to the rest of the electronics are also bulky,
			especially considering that only a single signal wire is used within each
			8-core cable.
			
			Due to time limitations, the end-stops were only used for positioning.
			Though not used to their full capability, the end-stops provide a valuable
			improvement to the printer's operation.  Other uses for the sensors and a
			fix for the ambient lighting issues are described in
			\S\ref{sec:future_endstop}.
		
	\section{Future Work}
		
		There are many possible avenues of future work which would either strongly
		complement the project or build directly on the system. The most interesting
		of these possibilities are presented below.
		
		\subsection{Network Interface}
			
			\label{sec:future_network}
			
			The interface used to interact with the printer is an important part of
			the system as it is both user-facing and performance critical.
			Unfortunately, while adequately performant, the interface built falls
			short in other areas and these issues are deserving of further work.
			
			\subsubsection{G-code Interface}
				
				No job control, security or multiple-user support is provided by the
				G-code interface of the printer. Each of these shortcomings restricts
				the printer to networks of trusted users who are able to collaborate to
				manually schedule print jobs. Future work might build upon conventional
				printer interfaces, for example integrating with the central Unix
				printing system (CUPS), to provide a more polished user experience.
			
			\subsubsection{Network Stack}
				
				Due to the limitations and bugs in \uIP{} an alternative stack such as
				lwIP could be used \cite{lwip}. This would provide a higher level,
				FreeRTOS-integrated network interface and, with its better tested flow
				control features, would enable TCP to be used for G-code transmission.
				This improved interface could also allow changes to the network
				interface to be significantly cleaner to implement.
			
			\subsubsection{Web Interface}
				
				The Mbed is powerful enough to generate and host dynamic web pages as
				demonstrated by the FreeRTOS web server demo. A web application for
				control and monitoring of the printer via a web browser could make the
				printer significantly easier to use by not requiring specialist
				software.
		
		\subsection{Endstop Support}
			
			\label{sec:future_endstop}
			
			To eliminate problems caused by lighting, a more advanced system could be
			implemented.  The infra-red LEDs in the opto-interrupters could be rewired
			such that they are pulsed, the photo-transistor's signal is then checked
			by the microcontroller for the presence of these pulses. External light
			sources are unlikely to contain matching pulses and so the system can be
			sure of the origin of the light passing into the photo-transistor.
			
			As well as this, the end-stops could be used for safety to detect when the
			axis have moved outside their operating area unexpectedly. This could help
			prevent damage to the printer and ensure safer operation when G-code with
			unreachable coordinates are used or mechanical failures occur.
			
		\subsection{G-Code Support}
			
			The subset of G-code supported by the printer is limited to that produced
			by Skeinforge. To allow other tools to be used for G-code generation and
			also to allow more features of the printer to be exposed, the G-code
			support could be extended.
		
		\subsection{Firmware Improvements}
			
			\label{sec:future_firmware}
			
			Two major improvements could be made to the firmware to better account for
			the mechanical properties of the printer. These are outlined below.
			
			Complex stepper control systems take into account the properties of
			stepper motors and do not step at a constant rate.  Instead, the speed is
			increased and decreased gradually making use of the increased torque
			available at low speeds during acceleration and deceleration. Such
			behaviours unfortunately also require more precise control of the extruder
			as the amount of plastic required throughout each line segment would vary
			with the speed of movement.
			
			The heaters could potentially respond more appropriately if variable
			amounts of power could be supplied (rather than just `on' and `off'). This
			requires PWM support to be added and changes made to the PID controller.
			PID controllers for variable power systems often require complex additions
			to function correctly and require careful set up.
			
		\subsection{Mechanical Improvements}
			
			As well as the aspects focused on in this project, many improvements could
			be made by modifying the printer's mechanical components. This work is the
			focus of many hobbyists and organisations with improvements in further
			generations of the Makerbot design building on these ideas. Porting
			promising ideas from other printers could provide valuable improvements in
			performance complementing the improved control of the printer achieved in
			this project.
	
	\section{Final Conclusions}
		
		The limitations of the network interface are probably the most significant
		but, with some modest restrictions, are not fatal. Future work in this area
		could address these limitations and explore possibilities for network
		printing with 3D printers.
		
		Overall, the project has been a success with the performance and
		extensibility of the printer being improved. As shown in appendix
		\ref{sec:workflow}, the printer is usable and follows essentially the same
		usage pattern as other 3D printers.
