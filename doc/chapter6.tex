\chapter{Conclusions \& Future Work}
	
	\label{sec:conclusions}
	
	% XXX: Eeek
	In this chapter, the outcomes of the project are compared against the initial
	goals. Finally, future work based on things learnt from the project and the
	limitations of the system is suggested.
	
	\section{Project Goals}
		
		The three project aims have each been addressed within the project and in
		this section the degree to which they have been met is described.
		
		\subsection{Electronics}
			
			The new electronics functionally replace all the features of the old
			electronics and have proved reliable in testing.  With a single board the
			system is also a lot simpler than the previous three boards (featuring a
			custom communications protocol and extra microcontroller).  Though not
			quite as polished as a printed circuit board (PCB), it offers a reasonable
			amount of space for future expansion. The
			
			The solid state MOSFETs, used for heater control, have performed well with
			the silent operation notable over the previous system. They also allow the
			possibility of variable power control with only software changes.
			
			The only notable issue remaining with the electronics is its compatibility
			with certain ATX PSUs. Despite the simplicity of the fix, time was not
			available to implement and test the change. Future work should aim to
			address this issue.
		
		\subsection{Firmware}
			
			The firmware written for the Mbed proved to be an improvement over the
			previous system making more detailed prints possible. It is also far more
			modular in design making expansion and experimentation feasible. The print
			quality achieved by the printer is now consistent with the constraints of
			the hardware itself rather than inadequacies of the firmware.
			
			FreeRTOS proved a good choice for the operating system as it provided a
			reliable multi-process environment. It also didn't place any restrictions
			on development allowing straightforward code for interacting with the
			low-level features of the Mbed.
			
			\uIP{}, with the exception of the TCP flow control issue, was well suited
			to the project providing the features needed allowing straightforward
			development. As well as this, the API it enforces (requiring protothreads
			and protosockets) is likely to make future expansion of the network
			interface difficult due to it's heavy restrictions. Future changes to
			overcome these problems are discussed in \S\ref{sec:future_network}.
			
			The stepper and heater control systems were adequate for this project
			though relatively na\"ive. Possible future improvements are described in
			\S\ref{sec:future_firmware}.
		
		\subsection{End-stops}
			
			Though susceptible glitches under direct lighting, the end-stops developed
			functioned correctly and allowed automatic calibration of the axis
			positions as required. They also obey the standard interface used by other
			Makerbot end-stops meaning future modifications may make use of standard
			designs.
			
			The need to build a control circuit resulted in another circuit board on
			the printer reducing some of the simplicity gained from simplifying the
			main electronics. The CAT-5 cables used to connect to the rest of the
			electronics are also bulky, especially considering that only a single
			signal wire is contained within each cable.
			
			Due to time limitations, the end-stops are not used except for
			positioning. There are other uses for the sensors which are described in
			\S\ref{sec:future_endstop}.
		
	\section{Future Work}
		
		There are many possible avenues of future work which would either strongly
		complement the results of this project or build directly on the system. The
		most interesting of these possibilities are presented below.
		
		\subsection{Network Interface}
			
			\label{sec:future_network}
			
			The interface with the printer is an important part of the system.
			Unfortunately, the interface built falls short in some areas and these
			issues are deserving of further work.
			
			\subsubsection{G-code Interface}
				
				No job control, security or multiple-user support is provided by the
				G-code interface of the printer. Each of these problems restricts the
				printer to networks of trusted users who are able to manually
				collaborate when scheduling print jobs. Future work might build upon
				conventional printer interfaces, for example integrating with CUPS, to
				extend the work already done in this area.
			
			\subsubsection{Network Stack}
				
				Due to the limitations and bugs in \uIP{} an alternative stack such as
				lwIP could be used \cite{lwip}. This would provide a higher level,
				FreeRTOS-integrated network interface and enable TCP to be used for
				G-code transmission. This improved interface could also make other
				improvements to the network interface significantly cleaner.
			
			\subsubsection{Web Interface}
				
				The Mbed is powerful enough to generate and host dynamic web pages. A
				web application for control and monitoring of the printer via a web
				interface could make the printer significantly easier to use.
		
		\subsection{Endstop Support}
			
			\label{sec:future_endstop}
			
			To eliminate problems caused by lighting, a more complex system could be
			implemented.  The infra-red LEDs in the opto-interrupters could be rewired
			such that they can be pulsed by the microcontroller. The
			photo-transistor's signal is then checked for the presence of these
			pulses. External light sources are unlikely to contain the same pulses and
			so the system can be sure of the origin of the light passing onto the
			photo-transistor.
			
			As well as this, the end-stops could be used for safety to detect when the
			axis have moved outside their operating area unexpectedly. This could help
			prevent damage to the printer when G-code with unreachable coordinates are
			used.
			
		\subsection{G-Code Support}
			
			The subset of G-code supported by the printer is limited to that produced
			by Skeinforge. To allow other tools to be used for G-code generation and
			also to allow more features of the printer to be exposed, the G-code
			support could be extended.
		
		\subsection{Firmware Improvements}
			
			\label{sec:future_firmware}
			
			% TODO: Cite motion compensation
			
			Complex stepper control systems don't often step at a constant rate.
			Instead the speed is increased and decreased gradually which takes
			advantage of stepper motors' increased talk at low speeds. Such a system
			would also require more precise control of the extruder as the amount of
			plastic required would vary throughout each line segment.
			
			% TODO: Cite PID hardness with variable power
			
			The heaters could potentially respond more appropriately if variable
			amounts of power could be supplied (rather than just `on' and `off'). This
			requires PWM support to be added to allow the heater to be controlled
			variably. PID controllers often require complex additions to enable
			control of such systems and require careful set up.
			
		\subsection{Mechanical Improvements}
			
			As well as the aspects focused on in this project, many improvements can
			be made by modifying the printer's mechanical components. This work is the
			focus of many hobbyists and organisations with improvements in further
			generations of Makerbot being made. Porting promising ideas from other
			printers could provide valuable improvements in performance.
			
			% XXX: Expand?
